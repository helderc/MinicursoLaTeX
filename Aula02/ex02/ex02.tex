\documentclass{article}

\usepackage[latin1]{inputenc}
\usepackage[brazil]{babel}
\usepackage{lipsum}

\begin{document}

\section{Ambiente de Listas: Itemize}
O exemplo do ambiente de listas Itemize...

\begin{itemize}
	\item Primeira linha da minha lista
	\item Segunda linha usando itemize
\item sdklfjsld
		\begin{itemize}
			\item Linha 1 do subnivel.
			\item Linha 2 do subnivel.
			\begin {itemize}
				\item Subsubnivel!
			\end{itemize}
		\end{itemize}
	\item Terceira linha!
\end{itemize}




\section{Ambiente de Listas: Enumerate}
Exemplo de lista utilizando o Enumerate.

\begin{enumerate}
	\item Minha primeira linha...
		\begin{enumerate}
			\item Subitem 1!
			\item Subitem 2!
			\begin{enumerate}
				\item Subsubitem.
			\end{enumerate}
			\item Subitem 3!
		\end{enumerate}
	\item Segunda linha.
	\item Terceira.
\end{enumerate}



\section{Ambiente de Listas: Description}
Ambiente de listas Description.

\begin{description}
	\item[um] descri��o de dois.
	\begin{description}
		\item[sub1] aqui � um subnivel!
	\end{description}
	\item[dois] descri��o de tr�s.
\end{description}


\end{document}