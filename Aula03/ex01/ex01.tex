\documentclass{article}

\usepackage[brazil]{babel}
\usepackage[latin1]{inputenc}
\usepackage{graphicx}
\usepackage{xcolor}


\begin{document}

\begin{figure}[!htb]
	\label{figura_01}
	\begin{center}
		\includegraphics{logo.pdf}	 % figura 01
	\end{center}
	\caption{Legenda que ir� aparacer no texto.}
\end{figure}


\begin{figure}[htbp]
	\label{figura_02}
	\includegraphics[width=12cm]{logo.pdf}	 % % figura 02
	\caption{Legenda que ir� aparacer no texto.}
\end{figure}


\begin{figure}
	\label{figura_03}
	\includegraphics[width=1in]{logo.pdf}  % figura 03
	\caption{Voc� reconhece esta imagem?}
\end{figure}


\begin{figure}
	\includegraphics[angle=60, width=1in]{logo.pdf}		% figura 04
	\caption{Voc� reconhece esta imagem?}
\end{figure}


\begin{figure}
  	\begin{center}
		\includegraphics[scale = 0.30]{logo.pdf}		% figura 05
	\end{center}
	\caption{Esta bela figura sofreu uma opera��o de escala de um fator igual a $0.3$.}
\end{figure}

\end{document}

