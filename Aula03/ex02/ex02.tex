\documentclass{article}

\usepackage[latin1]{inputenc}
\usepackage[brazil]{babel}
\usepackage{hyperref}

\begin{document}

\tableofcontents


\section{Exemplo N�O eficiente!}
Aqui usaremos a bibliografia inclusa no pr�prio .tex. Para ``citar'' um autor,
basta usar \verb"\cite{autor}" como em: \cite{baumgart} ou poderemos \cite{berg}.

Lembre-se !!!! 

� necess�rio compilar o .tex v�rias vezes para que as refer�ncias
sejam ``linkadas''.







%\bibliographystyle{plain}

\begin{thebibliography}{99}

\bibitem{baumgart} BAUMGART, B.G. {\bf Geometric modeling for computer vision}. 1974. 463. Relat\'orio t\'ecnico - Stanford Artificial Intelligence Laboratory, Computer Sciences Department, Stanford University, CA.

\bibitem{baumgart2} BAUMGART, B.G. {\bf Windge-edge polyhedron representation}. 1972. 320. Relat\'orio T\'ecnico - Computer Science Department, Standford University, Palo Alto, CA.

\bibitem{berg} BERG, M. K.; OVERMARS, M.V.; SCHWARZKOPF, M. O. {\bf Computational geometry}: algorithms and applications. Springer-Verlag, 1977.

\bibitem{bowyer} BOWYER, A.; WOODWARK, J. {\bf A programmer's geometry}. Butterworths, 1983.

\end{thebibliography}


\end{document}