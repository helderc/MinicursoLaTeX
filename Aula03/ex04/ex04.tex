\documentclass{article}


\usepackage[brazil]{babel}
\usepackage[latin1]{inputenc}

\usepackage{amsmath}


\begin{document}

\section{Regra de L'Hospital}

As regras de L'Hospital, que vamos enunciar a seguir e que cujas demonstra��es 
s�o deixadas para o final da se��o, aplicam-se a c�lculos de limites que 
apresentam indetermina��es dos tipos $\frac{0}{0}$ e $\frac{\infty}{\infty}$.
\\ \\
A primeira regra diz: Sejam $f$ e $g$ deriv�veis em $]p-r, p[$ e em 
$]p,\ p+r[\ (r > 0)$, com $g'(x) \neq 0$ para $0 < |x-p| < r$. 
Nessas condi��es, se

\[
	\lim_{x\rightarrow p}{f(x)}\ =\ 0, \quad 
	\lim_{x\rightarrow p}{g(x)}\ =\ 0
\]

e se $\lim\limits_{x\rightarrow p}{\frac{f'(x)}{g'(x)}}$ existir 
(finito ou infinito), ent�o $\lim\limits_{x\rightarrow p}{\frac{f(x)}{g(x)}}$
existir� e

\[
	\lim_{x\rightarrow p}{\frac{f(x)}{g(x)}}\ =\ 
	\lim_{x\rightarrow p}{\frac{f'(x)}{g'(x)}}.
\]

\end{document}

