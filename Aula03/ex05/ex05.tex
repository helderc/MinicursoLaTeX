\documentclass[leqno]{article}
% fleqn - as equa��es ficam a esquerda
% leqno - a numera��o fica a esquerda

\usepackage[brazil]{babel}
\usepackage[latin1]{inputenc}

\begin{document}

\section{Ambiente de Equa��es}

\subsection{S�rie de Taylor}

Usando o {\verb"\begin{displaymath} ... \end{displaymath}"}
\begin{displaymath}
	T(x) = \sum_{n = 0}^{\infty}{a_n(x-a)^n}
\end{displaymath}


Agora usando {\verb"\[ ... \]"}
\[
	T(x) = \sum_{n = 0}^{\infty}{a_n(x-a)^n}
\]



\subsection{Fun��o W de Lambert}

Equa��o com {\verb"\begin{equation} ... \end{equation}"}
\begin{equation}
W_0(x) = \sum_{n=1}^{\infty}{\frac{(-n)^{n-1}}{n!}x^n}\ para\ |x|<\frac{1}{e}
\end{equation}




\end{document}