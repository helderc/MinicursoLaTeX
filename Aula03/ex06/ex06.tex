\documentclass{article}

\usepackage[brazil]{babel}
\usepackage[latin1]{inputenc}

\begin{document}

\section{Equa��es em M�ltiplas Linhas}

\vspace{2cm}

%%%%% Lembrar o uso do &
%%%%% nonumber - n�o numera a equa��o

Usando {\verb"\begin{eqnarray} ... \end{eqnarray}"}
\center {$\vdots$}

\begin{eqnarray}
(x, y, z) & = & x(1, 0, 1) + y(0, 1, -1) + (-x+y+z)(0, 0, 1) \\
T(x, y, z) & = & xT(1, 0, 1) + yT(0, 1, -1) + (-x+y+z)T(0, 0, 1) \nonumber \\
T(x, y, z) & = & x(0, 0, 0, 0) + y(0, 0, 0, 0) + (-x+y+z)(1, 0, -1, 0) \\
T(x, y, z) & = & (-x+y+z, 0, x-y-z, 0)
\end{eqnarray}

\vspace{2cm}

\flushleft {Agora usando {\verb"\begin{eqnarray*} ... \end{eqnarray*}"}}
\center {$\vdots$}

% repare na falta do & 
\begin{eqnarray*}
(x, y, z) = x(1, 0, 1) + y(0, 1, -1) + (-x+y+z)(0, 0, 1) \\
T(x, y, z) = xT(1, 0, 1) + yT(0, 1, -1) + (-x+y+z)T(0, 0, 1) \\
T(x, y, z) = x(0, 0, 0, 0) + y(0, 0, 0, 0) + (-x+y+z)(1, 0, -1, 0) \\
T(x, y, z) = (-x+y+z, 0, x-y-z, 0)
\end{eqnarray*}



\end{document}