%%% -*-LaTeX-*-
%%% demo-colunas.tex -- exemplo de colunas com tabbing environment.
%%% $Id: demo-colunas.tex,v 1.3 2000/08/02 01:17:42 jessen Exp $

\section{Formatando em colunas}
\index{colunas|(}%

Alguns exemplos do \eng{environment} \environ{tabbing}:

%%% tab stop are set with the \= command
%%% \> moves to the next tab stop
\begin{tabbing}
  Coluna 1      \=Coluna 2      \=Coluna3       \=Coluna 4 \\

  Col1     \>     Col2      \>    Col3      \>    Col 4 \\
  Col1     \>     Col2      \>    Col3      \>    Col 4 \\
  Col1     \>     Col2      \>    Col3      \>    Col 4 \\
  Col1     \>     Col2      \>    Col3      \>    Col 4
\end{tabbing}

Agora definindo 8 colunas, mas pulando os \eng{tab stops} de dois em
dois:

\begin{tabbing}
  Coluna 1      \=Coluna 2      \=Coluna3       \=Coluna 4
  \=Coluna 5    \=Coluna 6      \=Coluna7       \=Coluna 8\\

  Col1     \>\>     Col2      \>\>    Col3      \>\>    Col 4 \\
  Col1     \>\>     Col2      \>\>    Col3      \>\>    Col 4 \\
  Col1     \>\>     Col2      \>\>    Col3      \>\>    Col 4 \\
  Col1     \>\>     Col2      \>\>    Col3      \>\>    Col 4
\end{tabbing}

O mesmo que o anterior, mas usando \command{kill} na primeira linha
para n�o produzir nenhum \eng{output}, apenas setar os \eng{tab
  stops}.

\begin{tabbing}
  Coluna 1      \=Coluna 2      \=Coluna3       \=Coluna 4
  \=Coluna 5    \=Coluna 6      \=Coluna7       \=Coluna 8\kill

  Coluna 1 \>\>     Coluna 2  \>\>    Coluna3   \>\>     Coluna 4\\
  Col1     \>\>     Col2      \>\>    Col3      \>\>    Col 4 \\
  Col1     \>\>     Col2      \>\>    Col3      \>\>    Col 4 \\
  Col1     \>\>     Col2      \>\>    Col3      \>\>    Col 4 \\
  Col1     \>\>     Col2      \>\>    Col3      \>\>    Col 4
\end{tabbing}
\index{colunas|)}%

%%% Used by GNU Emacs and AUC TeX.
%%% Local Variables:
%%% TeX-master: "demo.tex"
%%% TeX-auto-save: t
%%% End:

%%% demo-colunas.tex ends here.
