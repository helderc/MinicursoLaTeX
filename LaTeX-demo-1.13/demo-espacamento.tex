%%% -*-LaTeX-*-
%%% demo-espacamento.tex -- exemplo de espa�amento vertical e horizontal.
%%% $Id: demo-espacamento.tex,v 1.4 2000/08/02 01:17:42 jessen Exp $

\section{Espa�amento}
\index{espacamento@espa�amento|(}%

\subsection{Espa�amento Vertical}
\index{espacamento@espa�amento!vertical}%

Espa�amento vertical pode ser feito com o comando \command{vspace}.
\vspace{1cm}
Aqui temos \unidade{1}{cm} entre esta linha e a de cima.

\subsection{Espa�amento Horizontal}
\index{espacamento@espa�amento!horizontal}%
Espa�os em branco podem ser feitos com o comando \command{hspace}.

Por exemplo, aqui temos\hspace{3cm}\unidade{3}{cm} em branco.  Espa�os
tamb�m podem ser negativos, servindo como um \eng{backspace}, como
aqui\hspace{-1cm}XXX.

\subsubsection{hfill}

\command{hfill} � um caso interessante de \command{hspace}, onde o espa�o
em branco � maximizado:

Exemplo \hfill{} Exemplo.

Exemplo \hfill{} Exemplo \hfill{} Exemplo.

Exemplo \hfill{} Exemplo \hfill{} Exemplo \hfill{} Exemplo.

\subsubsection{dotfill e hrulefill}

\command{dotfill} e \command{hrulefill} funcionam de maneira an�loga, mas
em vez de espa�os produzem pontos e uma linha horizontal, como no
exemplo abaixo:

Exemplo \dotfill{} Exemplo \dotfill{} Exemplo.

Exemplo \hrulefill{} Exemplo \hrulefill{} Exemplo.

\index{espacamento@espa�amento|)}%

%%% Used by GNU Emacs and AUC TeX.
%%% Local Variables:
%%% TeX-master: "demo.tex"
%%% TeX-auto-save: t
%%% End:

%%% demo-espacamento.tex ends here.
