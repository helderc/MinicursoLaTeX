%%% -*-LaTeX-*-
%%% demo-espaco.tex -- exemplo de diversos tipos de espa�o.
%%% $Id: demo-espaco.tex,v 1.3 2000/07/26 04:44:35 jessen Exp $

\section{Espa�o em Branco}
\index{espaco em branco@espa�o em branco}%

\index{espaco em branco@espa�o em branco!pequeno}%
\command{,} produz um pequeno espa�o, como em ``foo `bar'\,''.

\index{espaco em branco@espa�o em branco!entre palavras}%
\index{" @\texttt{\symbol{'134}}\verb*# #}%
\index{comando!" @\texttt{\symbol{'134}}\verb*# #}%

\verb*!\ ! produz um espa�o entre palavras, como em Sr.\ bar.

\index{espaco em branco@espa�o em branco!sem quebra de linha}%
\verb!~! produz um espa�o entre palavras, onde quebra de linha n�o
pode ocorrer, como em n�mero~1.

\LaTeX{} sempre assume que um ponto termina uma senten�a, a n�o ser que
o ponto venha logo ap�s um letra mai�scula.

\index{espaco em branco@espa�o em branco!sentence-ending@
\eng{sentence-ending}}%
\command{@} produz um espa�o ``\eng{sentence-ending}'', nos casos em que
um ponto deve terminar uma senten�a, independente do caracter que
venha antes, como vitamina C\@.  � isso.

%%% Used by GNU Emacs and AUC TeX.
%%% Local Variables:
%%% TeX-master: "demo.tex"
%%% TeX-auto-save: t
%%% End:

%%% demo-espaco.tex ends here.
