%%% -*-LaTeX-*-
%%% demo-length.tex -- exemplo de lengths.
%%% $Id: demo-length.tex,v 1.2 1998/11/29 03:22:53 jessen Exp $

\section{\eng{Lengths}}

\eng{Length} � uma medida de dist�ncia, positiva ou negativa,
representado por um n�mero seguido de uma unidade.  Algumas das
unidades mais comuns s�o mostradas na Tab.~\ref{tab:units}.

\begin{table}[htbp]
  \centering
  \begin{tabular}{|c|r|r|}
    \hline
    Unidade     &       Nome & Equival�ncia \\
    \hline\hline
    \texttt{pt} &\eng{Point} & $\unidade{1}{pt} = \unidade{1/72.27}{in}$ \\
    \texttt{mm} &  Mil�metro & $\unidade{1}{mm} = \unidade{2.845}{pt}$ \\
    \texttt{pc} & \eng{Pica} & $\unidade{1}{pc} = \unidade{12}{pt}$ \\
    \texttt{cm} & Cent�metro & $\unidade{1}{cm} = \unidade{10}{mm}$ \\
    \texttt{in} &   Polegada & $\unidade{1}{in} = \unidade{25.4}{mm}$ \\
    \texttt{ex} &         Ex & Altura de um ``x'' \\
    \texttt{em} &         Em & Largura de um ``M'' \\
    \hline
  \end{tabular}
  \caption{Algumas das Unidades de Dist�ncia usadas pelo \LaTeX}
  \label{tab:units}
\end{table}

Al�m da possibilidade de usar medidas diretamente (como
\unidade{10}{cm}, \unidade{1}{ex}, etc.), \LaTeX{} tamb�m define
\eng{length commands}, isto �, comandos cujos valores s�o
\eng{lengths}.  Por exemplo, \lengthname{parindent} cont�m a medida da
indenta��o usada no come�o de um par�grafo.  Para uma descri��o de
todos os \eng{length commands} e como eles afetam o estilo de um
documento \LaTeX{} consulte~\cite[Ap�ndice~C]{IB-D883079}.

%%%%%%%%%%%%%%%%%%%%%%%%%%%%%%%%%%%%%%%%%%%%%%%%%%%%%%%%%%%%%

\subsection{Mostrando o Valor de um \eng{Length}}
\index{lengths@\eng{lengths}!mostrando}%

O valor de um \eng{length command} pode ser mostrado com o comando
\command{the} seguido do nome da medida.  Por exemplo,
\lengthname{parindent} = \the\parindent.

%%%%%%%%%%%%%%%%%%%%%%%%%%%%%%%%%%%%%%%%%%%%%%%%%%%%%%%%%%%%%

\subsection{Alterando o Valor de um \eng{Length}}
\index{lengths@\eng{lengths}!alterando}%

O valor de um \eng{length command} pode ser alterado com o comando
\command{setlength}.  Por exemplo, para zerar \lengthname{parindent}
temos:

\begin{codeverbatim}
\setlength{\parindent}{0pt}
\end{codeverbatim}

Note que um \eng{length command} pode ser definido em fun��o de outro.
No exemplo abaixo \lengthname{abovecaptionskip} (espa�o acima do
\eng{caption} e sua figura/tabela) � definido como duas vezes
seus valor original:

\begin{codeverbatim}
\setlength{\abovecaptionskip}{2\abovecaptionskip}
\end{codeverbatim}

Tamb�m � poss�vel somar um valor a um \eng{length command} com o
comando \command{addtolength}, como mostrado no exemplo abaixo, onde
\lengthname{parindent} � aumentado em \unidade{10}{pt}:

\begin{codeverbatim}
\addtolength{\parindent}{10pt}
\end{codeverbatim}

%%%%%%%%%%%%%%%%%%%%%%%%%%%%%%%%%%%%%%%%%%%%%%%%%%%%%%%%%%%%%

\subsection{Criando um novo \eng{Length}}
\index{lengths@\eng{lengths}!criando}%

O comando \command{newlength} define um novo \eng{length command}, com
valor inicial igual a 0, como mostrado no exemplo abaixo:

\begin{codeverbatim}
\newlength{\foo}
\end{codeverbatim}

Caso o \eng{length command} j� exista um erro � reportado.

%%% Used by GNU Emacs and AUC TeX.
%%% Local Variables:
%%% TeX-master: "demo.tex"
%%% TeX-auto-save: t
%%% End:

%%% demo-length.tex ends here.
