%%% -*-LaTeX-*-
%%% demo-poesia.tex -- exemplo de verse environment.
%%% $Id: demo-poesia.tex,v 1.3 2000/07/26 04:44:35 jessen Exp $

\section{Poesia}
\index{poesia}%
\index{verse@\eng{verse}|see{poesia}}%

\begin{comment}
\begin{verse}
  Oh beb�, beb�, beb� \\
  todas as noites eu tenho os azuis \\
  Oh beb�, beb�, beb� \\
  todas as noites eu tenho os azuis \\
  Sim, mam�e bonitinha \\
  eu tenho os azuis e isso n�o � bom.

  Eu fui pra S�o Lu�s, beb� \\
  e os azuis foram bem atr�s \\
  Eu disse que fui pra S�o Lu�s, beb� \\
  e os azuis foram bem atr�s \\
  Se voc� n�o voltar pra mim, mam�e bonitinha \\
  eu penso que vou perder minha mente.
\end{verse}
\end{comment}

\index{poesia!Mario Quintana@M�rio Quintana}%
\index{Mario Quintana@M�rio Quintana}%
\begin{verse}
  \textbf{Pequeno Poema Did�tico}\\
  (M�rio Quintana)\\[10pt]

  O tempo � indivis�vel. Dize, \\
  qual o sentido do calend�rio? \\
  Tombam as folhas e fica a �rvore, \\
  contra o vento incerto e v�rio.

  A vida � indivis�vel. Mesmo \\
  a que se julga mais dispersa \\
  e pertence a um eterno di�logo \\
  a mais inconsequente conversa.

  Todos os poemas s�o um mesmo poema, \\
  todos os porres s�o o mesmo porre, \\
  n�o � de uma vez que se morre\ldots \\
  Todas as  horas s�o horas extremas!
\end{verse}

%%% Used by GNU Emacs and AUC TeX.
%%% Local Variables:
%%% TeX-master: "demo.tex"
%%% TeX-auto-save: t
%%% End:

%%% demo-poesia.tex ends here.
