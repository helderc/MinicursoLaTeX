%%% -*-LaTeX-*-
%%% demo-quimica.tex -- exemplo de rea��es qu�micas
%%% $Id: demo-quimica.tex,v 1.6 2000/08/01 00:32:55 jessen Exp $

\section{Qu�mica}
\index{quimica@qu�mica!exemplos|(}%
\index{reacoes quimicas@rea��es qu�micas|see{qu�mica}}%

Esta se��o mostra alguns exemplos de rea��es qu�micas usando os
\eng{environments} \environ{equation} e \environ{eqnarray}.

%%%%%%%%%%%%%%%%%%%%%%%%%%%%%%%%%%%%%%%%%%%%%%%%%%%%%%%%%%%%

\index{acentuacao@acentua��o!em math mode@em \eng{math mode}}%
\index{acentuacao@acentua��o!em math mode@em \eng{math mode}!usando
text@usando \comm{text}}%
\begin{equation}
  \underbrace{A+B+\cdots}_{\text{Reagentes}} \rightleftharpoons
  \underbrace{X,Y,\cdots}_{\text{Intermedi�rios}} \rightleftharpoons
  \underbrace{P+Q+\cdots}_{\text{Produtos}}
\end{equation}

%%% Equa��o id�ntica a anterior, mas gera acentos em math mode
%%% diretamente, sem o uso da macro \text.
% \begin{equation}
%   \underbrace{A+B+\cdots}_{\mathrm{Reagentes}} \rightleftharpoons
%   \underbrace{X,Y,\cdots}_{\mathrm{Intermedi\acute{a}rios}}
%   \rightleftharpoons
%   \underbrace{P+Q+\cdots}_{\mathrm{Produtos}}
% \end{equation}

%%%%%%%%%%%%%%%%%%%%%%%%%%%%%%%%%%%%%%%%%%%%%%%%%%%%%%%%%%%%

\begin{eqnarray}
  \left.
    \begin{array}{l}
      A+Y \rightarrow X + P \nonumber \\
      X+Y \rightarrow 2P \nonumber
    \end{array}
  \right\}
  \mathrm{Processo~A} \\
  \left.
    \begin{array}{l}
      A+X \rightarrow 2X +Z \nonumber \\
      2X \rightarrow A + P \nonumber
    \end{array}
  \right\}
  \mathrm{Processo~B} \\
  \left.
    \begin{array}{l}
      Z \rightarrow \mathit{f} \cdot Y
    \end{array}
  \right\}
  \mathrm{Processo~C}
\end{eqnarray}

%%%%%%%%%%%%%%%%%%%%%%%%%%%%%%%%%%%%%%%%%%%%%%%%%%%%%%%%%%%%

\begin{equation}
  \mathrm{HBrO}_{2}+\mathrm{BrO}^{-}_{3}+3\mathrm{H}^{+}+2\mathrm{Fe}(II)
  \rightleftharpoons
  2\mathrm{Fe}(III) + 2\mathrm{HBrO}_{2}+\mathrm{H}_{2}\mathrm{O}
\end{equation}

%%%%%%%%%%%%%%%%%%%%%%%%%%%%%%%%%%%%%%%%%%%%%%%%%%%%%%%%%%%%

\begin{equation}
  \mathrm{Zn}_{(s)} + {\mathrm{H}_{2}\mathrm{SO}_{4}}_{(sol)} =
  {\mathrm{ZnSO}_{4}}_{(sol)} + {\mathrm{H}_{2}}_{(g)}
\end{equation}

%%%%%%%%%%%%%%%%%%%%%%%%%%%%%%%%%%%%%%%%%%%%%%%%%%%%%%%%%%%%

\begin{equation}
  \mathrm{C}_{10}\mathrm{H}_{8 (s)} + 12\,\mathrm{O}_{2 (g)} =
  10\,\mathrm{CO}_{2 (g)} + 4\,\mathrm{H}_{2}\mathrm{O}_{(l)}
\end{equation}

%%%%%%%%%%%%%%%%%%%%%%%%%%%%%%%%%%%%%%%%%%%%%%%%%%%%%%%%%%%%

\begin{equation}
  \mathrm{H}_{2}\mathrm{O}_{(s)} = \mathrm{H}_{2}\mathrm{O}_{(l)}
  \Delta \mathrm{H}_{273} = 1438\, \mathrm{cal} \cdot \mathrm{mol}^{-1}
\end{equation}

%%%%%%%%%%%%%%%%%%%%%%%%%%%%%%%%%%%%%%%%%%%%%%%%%%%%%%%%%%%%

\begin{equation}
  \mathrm{H}_{2}\mathrm{O}_{(l)} =
  \mathrm{H}_{2}\mathrm{O}_{(\mathrm{g}, \ 0.0313\,\mathrm{atm})}
  \Delta \mathrm{H} = 10514\,\mathrm{cal} \cdot \mathrm{mol}^{-1}
\end{equation}

%%%%%%%%%%%%%%%%%%%%%%%%%%%%%%%%%%%%%%%%%%%%%%%%%%%%%%%%%%%%

\begin{equation}
  \mathrm{C}_{p} = 10.0 + 4.84 \times 10^{-3}\,\mathrm{T} -
  0.1080 \times 10^{-6}\mathrm{T}^{-2}\, \mathrm{cal}
  \cdot \mathrm{mol}^{-1} \cdot \mathrm{K}^{-1}
\end{equation}

%%%%%%%%%%%%%%%%%%%%%%%%%%%%%%%%%%%%%%%%%%%%%%%%%%%%%%%%%%%%

\begin{eqnarray}
  R & = & 0.0820569 \, \mathrm{atm} \cdot \mathrm{mol}^{-1}
  \cdot \mathrm{K}^{-1} \nonumber \\
  & = & 8.31441 \, \mathrm{J} \cdot \mathrm{mol}^{-1}
  \cdot \mathrm{K}^{-1} \nonumber \\
  & = & 1.98719 \, \mathrm{cal} \cdot \mathrm{mol}^{-1}
  \cdot \mathrm{K}^{-1}
\end{eqnarray}

%%%%%%%%%%%%%%%%%%%%%%%%%%%%%%%%%%%%%%%%%%%%%%%%%%%%%%%%%%%%

\begin{equation}
  K_{e}= \frac{R T_{o}^{2} M_{1}}
  {1000 \,\Delta \mathrm{H}_{e}} = 2.16
\end{equation}

%%%%%%%%%%%%%%%%%%%%%%%%%%%%%%%%%%%%%%%%%%%%%%%%%%%%%%%%%%%%

\begin{equation}
  \nu = - \frac{d[\mathrm{AAS}]}{dt} =
  - \frac{d[\mathrm{OH}^{-}]}{dt} =
  \frac{d[\mathrm{AS}]}{dt} = \frac{d[\mathrm{Ac}^{-}]}{dt}
\end{equation}

%%%%%%%%%%%%%%%%%%%%%%%%%%%%%%%%%%%%%%%%%%%%%%%%%%%%%%%%%%%%

\begin{equation}
  \mathrm{AAS} + \mathrm{OH}^{-} \rightleftharpoons
  [\mathrm{HO} \cdots Salic \cdots \mathrm{OAc}]
\end{equation}

%%%%%%%%%%%%%%%%%%%%%%%%%%%%%%%%%%%%%%%%%%%%%%%%%%%%%%%%%%%%

\begin{equation}
  [\mathrm{AAS}]_{t} \propto (A_{\infty} - A_{t}) = \mathcal{A}_{t}
\end{equation}

%%%%%%%%%%%%%%%%%%%%%%%%%%%%%%%%%%%%%%%%%%%%%%%%%%%%%%%%%%%%

%%% From: asnd@erich.triumf.ca (Donald Arseneau)
%%% Subject: Re: Formatting chemical equations
%%% Newsgroups: comp.text.tex
%%% Date: 4 Jan 2000 16:25 PST

\newcommand\eqnhline{%
  \noalign{\nobreak\vskip-\ht\strutbox\vskip\dp\strutbox}%
  \multispan3{\hrulefill}\cr}

\begin{eqnarray}
  A + B & \rightarrow & C + D \nonumber \\
  C + D & \rightarrow & F \nonumber \\
  \eqnhline
  A + B & \rightarrow & F
\end{eqnarray}

%%%%%%%%%%%%%%%%%%%%%%%%%%%%%%%%%%%%%%%%%%%%%%%%%%%%%%%%%%%%

%A Eq.~\ref{eq:cte_de_equilibrio} � um exemplo de uso de constantes de
%equil�brio.
%\index{quimica@qu�mica!constantes de equil�brio}%

%\begin{equation}
%  \label{eq:cte_de_equilibrio}
%  A+B\overset{k_a}{\underset{k_b}{\rightleftharpoons}}C
%\end{equation}

%%%%%%%%%%%%%%%%%%%%%%%%%%%%%%%%%%%%%%%%%%%%%%%%%%%%%%%%%%%%

\index{quimica@qu�mica!is�topos}%
Exemplo de representa��o de um is�topo: $\nucl{16}{8}{O}$.

\index{quimica@qu�mica!exemplos|)}%

%%% Used by GNU Emacs and AUC TeX.
%%% Local Variables:
%%% TeX-master: "demo.tex"
%%% TeX-auto-save: t
%%% End:

%%% demo-quimica.tex ends here.
