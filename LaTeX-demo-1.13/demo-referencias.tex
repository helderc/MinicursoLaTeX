%%% -*-LaTeX-*-
%%% demo-referencias.tex -- exemplo de refer�ncias.
%%% $Id: demo-referencias.tex,v 1.4 2001/01/12 02:21:11 jessen Exp $

\section{Refer�ncias}
\index{referencias@refer�ncias!exemplo}%
\label{sec:referencias}

Refer�ncias s�o implementadas com os comandos \command{label},
\command{ref} e \command{pageref}.  Exemplos: Tab.~\ref{tab:cline} na
p�gina~\pageref{tab:cline}, Fig.~\ref{fig:gnu} na
p�gina~\pageref{fig:gnu}, Eq.~\ref{eq:sum} na p�gina~\pageref{eq:sum},
Teorema~\ref{theorem:chasles} na p�gina~\pageref{theorem:chasles},
item~\ref{list:bar} da p�gina~\pageref{list:bar}.

\subsection{Refer�ncias com o \eng{package} \pack{varioref}}
\index{referencias@refer�ncias!com package varioref@com o
\eng{package} \package{varioref}}%

O \eng{package} \package{varioref} define novos comandos:
\command{vref} e \command{vpageref}.  O comando \command{vref} �
similar ao comando \command{ref} mas adiciona uma refer�ncia adicional
da forma `na p�gina anterior', `na pr�xima p�gina' ou `na p�gina 100'
caso o \command{label} n�o esteja na mesma p�gina.

O comando \command{vpageref} � uma varia��o do comando
\command{pageref} mas que tamb�m leva em conta a proximidade do
\command{label} correspondente, da mesma forma que \command{vref}.

Exemplos: se��o~\vref{sec:referencias} \vpageref{sec:referencias},
Fig.~\vref{fig:gnuplot-example-2}, se��o~\vref{sec:typed-text},
Teorema~\vref{theorem:chasles}.

%%% Used by GNU Emacs and AUC TeX.
%%% Local Variables:
%%% TeX-master: "demo.tex"
%%% TeX-auto-save: t
%%% End:

%%% demo-referencias.tex ends here.
