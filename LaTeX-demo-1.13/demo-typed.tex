%%% -*-LaTeX-*-
%%% demo-typed.tex -- exemplo do environment vebatim, package
%%% alltt e package fancyvrb para inclus�o literal de material em
%%% documentos.
%%% $Id: demo-typed.tex,v 1.9 2001/01/12 02:25:41 jessen Exp $

%%%  Veja tamb�m:
%%%  fancyvrb CTAN:macros/latex/contrib/supported/fancyvrb/
%%%
%%%  Sophisticated handling of verbatim text including: verbatim commands in
%%%  footnotes; a variety of verbatim environments with many
%%%  parameters; ability to define new customized verbatim
%%%  environments; save and restore verbatim text and environments;
%%%  write and read files in verbatim mode; build ``example''
%%%  environments (showing both result and verbatim text).

\section{\eng{Typed text}}
\label{sec:typed-text}
\index{typed text@\eng{typed text}}%

Esta se��o mostra exemplos de inclus�o literal de material em
documentos.

%%%%%%%%%%%%%%%%%%%%%%%%%%%%%%%%%%%%%%%%%%%%%%%%%%%%%%%%%%%%
\subsection{\eng{Typed text}---sem espa�os}
\index{typed text@\eng{typed text}!sem espacos@sem espa�os}%
\index{verbatim@\eng{verbatim}|see{\eng{typed text}}}%

A forma mais simples � com o \eng{environment} \environ{verbatim},
como mostrado abaixo:

\begin{verbatim}
;;; tex-mode.el --- tex, latex, and slitex mode commands.

;; copyright (c) 1985, 86, 89, 92, 94, 95, 96, 1997
;;       free software foundation, inc.

;; maintainer: fsf
;; keywords: tex
\end{verbatim}

%%%%%%%%%%%%%%%%%%%%%%%%%%%%%%%%%%%%%%%%%%%%%%%%%%%%%%%%%%%%
\subsection{\eng{Typed text}---com espa�os}
\index{typed text@\eng{typed text}!com espacos@com espa�os}%

O \eng{environment} \environ{verbatim*} faz o mesmo, mas evidenciando
os espa�os em branco:

\begin{verbatim*}
;;; tex-mode.el --- tex, latex, and slitex mode commands.

;; copyright (c) 1985, 86, 89, 92, 94, 95, 96, 1997
;;       free software foundation, inc.

;; maintainer: fsf
;; keywords: tex
\end{verbatim*}

%%%%%%%%%%%%%%%%%%%%%%%%%%%%%%%%%%%%%%%%%%%%%%%%%%%%%%%%%%%%
\subsection{\eng{Typed text}---\eng{package} \pack{alltt}}
\label{subsec:alltt}
\index{typed text@\eng{typed text}!package alltt@\eng{package}
\pack{alltt}}%

O \eng{package} \package{alltt} define o \eng{environment}
\environ{alltt} que tem o mesmo efeito do \environ{verbatim}, mas
permite inclus�o de texto proveniente de arquivos.

\begin{alltt}\input{code/prog1.el}\end{alltt}

%%%%%%%%%%%%%%%%%%%%%%%%%%%%%%%%%%%%%%%%%%%%%%%%%%%%%%%%%%%%
\subsection{\eng{Typed text} com Moldura}
\label{subsec:verbframe}

� poss�vel incluir o material dentro de uma moldura atrav�s do
\eng{environment} \environ{Verbatim} com o par�metro
\verb!frame=single!.  Este \eng{environment} � definido pelo
\eng{package} \package{fancyvrb}.

%%% outros valores para frame= s�o: topline, botontline e lines.
\begin{Verbatim}[frame=single]
;;; tex-mode.el --- tex, latex, and slitex mode commands.

;; copyright (c) 1985, 86, 89, 92, 94, 95, 96, 1997
;;       free software foundation, inc.

;; maintainer: fsf
;; keywords: tex
\end{Verbatim}

%%%%%%%%%%%%%%%%%%%%%%%%%%%%%%%%%%%%%%%%%%%%%%%%%%%%%%%%%%%%
\subsection{\eng{Typed text} dentro de footnotes}
\label{subsec:verbfootnotes}

\index{footnote@\eng{footnote}!com verbatim@com \verb+verbatim+}%
\VerbatimFootnotes
Ap�s o uso do comando \command{VerbatimFootnotes}, material sem
formata��o tamb�m pode ser colocado dentro de \eng{footnotes}%
\footnote{\verb!_Exemplo de texto verbatim em footnotes_!}.

%%% Used by GNU Emacs and AUC TeX.
%%% Local Variables:
%%% TeX-master: "demo.tex"
%%% TeX-auto-save: t
%%% End:

%%% demo-typed.tex ends here.
