%%% -*-LaTeX-*-
%%% demo-url.tex -- exemplo de uso de URLs, paths e emails
%%% com o package url.
%%% $Id: demo-url.tex,v 1.4 2000/08/01 00:32:55 jessen Exp $

\section{URLs, \eng{Paths} e \eng{Emails}}
\label{sec:url}

O uso de URLs, \eng{paths} e \eng{emails} em documentos pode ser um
problema devido � dificuldade do \LaTeX{} em realizar quebra de linha
nestes casos.

%%%%%%%%%%%%%%%%%%%%%%%%%%%%%%%%%%%%%%%%%%%%%%%%%%%%%%%%%%%%
\subsection{URLs}
\index{urls}%
\index{urls!muito longas}%

O \eng{package} \package{url} define o comando \command{url}, uma
esp�cie de \command{verb} que permite quebra de linha e que pode ser
usado como argumento para outros comandos. (ao contr�rio do comando
\command{verb}).

Um exemplo de uma URL muito longa:
\url{http://www.cis.ohio-state.edu/hypertext/faq/usenet/radio/ham-radio/
digital-faq/faq.html}.

\index{urls!mudando o estilo}%
Estilo pode ser mudado com \command{urlstyle}, como por exemplo em:
\urlstyle{sf}
\url{http://ptolemy.eecs.berkeley.edu/~pino/Ptolemy/papers/96/dtmf_ict/}.
%%% volta ao estilo default, tt
\urlstyle{tt}

%%%%%%%%%%%%%%%%%%%%%%%%%%%%%%%%%%%%%%%%%%%%%%%%%%%%%%%%%%%%
\subsection{\eng{Paths}}
\index{paths@\eng{paths}}%

O \eng{package} \package{url} define tamb�m o comando \command{path},
�til para o uso de \eng{pathnames} muito longas, como por exemplo:
\path{/usr/local/src/ftp.win.tue.nl/tcp_wrappers/
tcp_wrappers_7.6.tar.gz}.

%%%%%%%%%%%%%%%%%%%%%%%%%%%%%%%%%%%%%%%%%%%%%%%%%%%%%%%%%%%%
\subsection{\eng{Mails}}
\index{email@\eng{email}!definicao de macro@defini��o de macro}%

Note que o \eng{package} \package{url} n�o define \command{email}.
Entretanto, pode-se defini-lo, por exemplo, como:

\begin{codeverbatim}
\newcommand\email{\begingroup \urlstyle{tt}\Url}
\end{codeverbatim}

Assim \eng{emails} podem ser usados, como em:
\email{alan@lxorguk.ukuu.org.uk} e
\email{jnweiger@immd4.informatik.uni-erlangen.de}.

%%% Used by GNU Emacs and AUC TeX.
%%% Local Variables:
%%% TeX-master: "demo.tex"
%%% TeX-auto-save: t
%%% End:

%%% demo-url.tex ends here.
