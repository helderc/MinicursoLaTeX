%%% $Id: report.tex,v 1.5 2001/01/20 02:01:06 jessen Exp $
\documentclass[a4paper,11pt]{report}

%%% define a macro \ifpdf para compila��o condicional -- PDF ou
%%% DVI/PS.
\newif\ifpdf
  \ifx\pdfoutput\undefined
  \pdffalse
\else
  \pdfoutput=1
  \pdftrue
\fi

%%% caracteres acentuados em ISO-8859-1
\usepackage[latin1]{inputenc}

%%% Suporte para gerar o documento em Portugu�s e Ingl�s
\usepackage[english,brazil]{babel}

%%% indenta primeiro par�grafo, estilo brasileiro.
\usepackage{indentfirst}

%%% �ndice Remissivo
\usepackage{makeidx}

\ifpdf
%%% somente na vers�o PDF

%%% Uso de Font Encoding T1 (simulado pelo package AE) -- importante
%%% para a correta hifeniza��o de palavras acentudas em Portugu�s.
\usepackage{ae}

%%% Para inclus�o de gr�ficos
\usepackage[pdftex]{graphicx}

%%% dimens�es do documento
\usepackage[pdftex]{geometry}
\geometry{a4paper,left=1in,right=1in,top=1cm,bottom=1cm}

\else
%%% somente na vers�o DVI/PS

%%% Uso de Font Encoding T1
\usepackage[T1]{fontenc}

%%% Para inclus�o de gr�ficos
\usepackage[dvips]{graphicx}

%%% dimens�es do documento
\usepackage[dvips]{geometry}
\geometry{a4paper,left=1in,right=1in,top=1in,bottom=1in}

\fi

\title{Esqueleto para um Documento \LaTeX:\\
  Classe \textsf{report}}

\author{Foob�rio da Silva\thanks{Foo de Oliveira Bar}\\
  \texttt{foo@bar.org}}

\date{\today\\
  Vers�o 0.1}

%%% Para a cria��o de �ndice
\makeindex

%%%% preamble ends here
\begin{document}

\maketitle
\thispagestyle{empty}

{%
  \selectlanguage{english}
  \begin{abstract}
    This is a \LaTeX{} simple document skeleton.  Use it as a base for
    your own documents.
  \end{abstract}
}

\begin{abstract}
  Este � um esqueleto de um documento simples em \LaTeX.  Use-o como
  base para seus pr�prios documentos.
\end{abstract}

%%% Sum�rio, lista de figuras e tabelas.
\tableofcontents
\listoffigures
\listoftables

\clearpage
\chapter{Nome do Primeiro Cap�tulo}
\section{Foo}
\index{Foo}%

Coloque sua se��o aqui.

\section{Bar}
\index{Bar}%

Coloque sua se��o aqui.

\subsection{Foobar}
\index{FooBar}%

Coloque sua subse��o aqui.

\chapter{Nome do Segundo Cap�tulo}
\section{Foobar}

\appendix
\chapter{Primeiro Ap�ndice}

Coloque seu ap�ndice aqui, conforme descrito em~\cite{IB-D883079}.

%%% Indice Remissivo
%%% quebra de p�gina opcional
\clearpage
\printindex

%%% Referencias
%%% quebra de p�gina opcional
\clearpage
\bibliography{template}
\bibliographystyle{plain}

\end{document}

%%% report.tex ends here.
