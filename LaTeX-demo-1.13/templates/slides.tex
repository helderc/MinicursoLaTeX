%%% $Id: slides.tex,v 1.5 2001/01/12 02:21:55 jessen Exp $
\documentclass[clock]{slides}

%%% caracteres acentuados em ISO-8859-1
\usepackage[latin1]{inputenc}

%%% A classe slides usa fontes muito grandes em sans serif para
%%% facilitar a leitura.  Estas fontes (neste tamanho) n�o est�o
%%% dispon�veis em encoding T1.  Usaremos o encoding default do LaTeX, OT1.

%%% Suporte para gerar o documento em Portugu�s e Ingl�s
\usepackage[brazil,english]{babel}

%%% Para fazer slides coloridos.  �til tamb�m para fazer overlays.
\usepackage{color}

%%% Imprime os slides
\onlyslides{1-99999}

%%% Imprime as notas para o orador
\onlynotes{1-99999}

%%%% preamble ends here

\begin{document}
\selectlanguage{brazil}

%%%%%%%%%%%%%%%%%%%%%%%%%%%%%%%%%%%%%%%%%%%%%%%%%%%%%%%%%%%%

%%% a classe slides a princ�pio n�o permite a inclus�o de figuras
%%% atrav�s do environment `figure'.  Isso pode ser contornado atrav�s
%%% do uso do package float, como mostrado abaixo:

%\usepackage{graphicx}
%\usepackage{float}
%
%\newfloat{figure}{H}{lof}% define a new `float' called `figure',
%                         % but make it non-floating with `H'ere placement
%
%\floatname{figure}{Figura}% the text used in the caption
%
%[...]
%
%  \begin{figure}
%    \centering
%    \includegraphics{figure.eps}
%    \caption{Figure in/on a slide.}
%  \end{figure}

%%%%%%%%%%%%%%%%%%%%%%%%%%%%%%%%%%%%%%%%%%%%%%%%%%%%%%%%%%%%
%% Slide 1: 2 minutos
\addtime{120}

\begin{slide}

  \begin{center}
    Esqueleto para um Documento \LaTeX:\\
    Classe slides
  \end{center}

  Qualquer comando pode ser usado, com exce��o de figuras, tabelas e
  comandos de quebra de p�gina.

\end{slide}

%%% notas para o orador
\begin{note}
  N�o esquecer de falar sobre isso e aquilo e aquilo outro.
\end{note}

%%%%%%%%%%%%%%%%%%%%%%%%%%%%%%%%%%%%%%%%%%%%%%%%%%%%%%%%%%%%
%% Slide 2: 4 minutos
\addtime{240}

\begin{slide}

  %%% \textcolor{white} s� aparece na vers�o PostScript,
  %%% n�o no dvi.

  Exemplo com overlays.  Foo e Bar aparecem no primeiro slide.
  \textcolor{white}{Foobar e Foobaz aparecem no segundo slide.}

  Foo

  Bar

  \textcolor{white}{Foobar}

  \textcolor{white}{Foobaz}

\end{slide}

\begin{note}
  N�o esquecer de falar sobre Foo e Bar.
\end{note}

%%%%%%%%%%%%%%%%%%%%%%%%%%%%%%%%%%%%%%%%%%%%%%%%%%%%%%%%%%%%
%%% Slide 3: 4 minutos
%%% Este slide vai por cima do anterior.

\addtime{240}
\begin{overlay}

  \textcolor{white}{Exemplo com overlays.  Foo e Bar aparecem no
    primeiro slide.}
  Foobar e Foobaz aparecem no segundo slide.

  \textcolor{white}{Foo}

  \textcolor{white}{Bar}

  Foobar

  Foobaz

\end{overlay}

\begin{note}
  N�o esquecer de falar sobre Foobar e Foobaz.
\end{note}

%%%%%%%%%%%%%%%%%%%%%%%%%%%%%%%%%%%%%%%%%%%%%%%%%%%%%%%%%%%%
%%% Slide 4: 3 minutos

\addtime{180}
\begin{slide}

  \begin{center}
    �ltimo slide: Exemplo com bullets
  \end{center}

  \begin{itemize}
  \item{Foo}
  \item{Bar}
  \item{Foobar}
  \item{FooBaz}
  \end{itemize}

\end{slide}

\begin{note}
  N�o esquecer de agradecer � plat�ia\ldots
\end{note}

%%%%%%%%%%%%%%%%%%%%%%%%%%%%%%%%%%%%%%%%%%%%%%%%%%%%%%%%%%%%

\end{document}

%%% slides.tex ends here.
