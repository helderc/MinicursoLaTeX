
\hspace{-7mm}{\Large{\bf Abstract }} \\ [6mm]

That work presents a theoretical discussion and the results obtained from the implementation of insert algorithms applied to the Delaunay triangulation problem in the plane. The representation of all of the phases of the constructive process is based on the \textit{modified winged edge} topological data structure.
 
Among all of the existent triangulations and associated to a same set of points, the Delaunay triangulation is that not only maximizes the smallest of the angles (MaxMin Criterium), but it guarantees under certain conditions, the unicity of the triangulation. In that sense, the triangulation of Delaunay associated to a set of points in $R^2$, presents characteristics of local and global regularity, reason for the which is indispensable for a wide group of applications, as for instance, terrain digital modelling and automatic finite elements meshes generation. The results previously obtained served as starting point so that a project associated to the diagram of Voronoi, that is the dual of the Delaunay triangulation, it could be obtained in an indirect way.


\vspace{15mm}

\hspace{-7mm}{\Large {\bf Keywords}}\\

\hspace{-7mm}
Delaunay Triangulation \\ [2mm]
Topological data structures \\ [2mm]
\textit{modified winged-edge} \\ [2mm]


 
%%% Local Variables: 
%%% mode: latex
%%% TeX-master: "tese"
%%% End: 
